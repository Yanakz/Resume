% !TEX TS-program = xelatex
% !TEX encoding = UTF-8 Unicode
% !Mode:: "TeX:UTF-8"

\documentclass{resume}
\usepackage{zh_CN-Adobefonts_external} % Simplified Chinese Support using external fonts (./fonts/zh_CN-Adobe/)
%\usepackage{zh_CN-Adobefonts_internal} % Simplified Chinese Support using system fonts
\usepackage{linespacing_fix} % disable extra space before next section
\usepackage{cite}
\renewcommand{\refname}{发表论文}
\begin{document}
\pagenumbering{gobble} % suppress displaying page number

\name{梁浩然}

% {E-mail}{mobilephone}{homepage}
% be careful of _ in emaill address
\contactInfo{lhr1988@126.com}{(+86) 15858178219}{应聘职位:网易游戏-人工智能工程师}
% {E-mail}{mobilephone}
% keep the last empty braces!
%\contactInfo{xxx@yuanbin.me}{(+86) 131-221-87xxx}{}
 
\section{\faGraduationCap\  教育背景}\small
\datedsubsection{\textbf{浙江工业大学}, 杭州}{2011 -- 至今}
\textit{博士}\ 控制科学与工程, 预计2017年3月毕业, \textit{研究方向:机器学习与模式识别}
\datedsubsection{\textbf{浙江工业大学}, 杭州}{2007 -- 2011}
\textit{学士}\ 计算机科学与技术

\section{\faUsers\ 实习/课题研究/项目经历}\small
\datedsubsection{\textbf{1. 新加坡国立大学}\  \textit{学术交流访问}}{2014年5月 -- 2015年9月}
进行视觉显著性相关方面的研究,利用计算机视觉相关算法设计视觉显著性模型

\datedsubsection{\textbf{2. 人脸图像合成算法}\ \textit{课题研究项目}}{2012年7月 -- 2013年7月}
\begin{onehalfspacing}
	内容:1. 利用二维人脸图像合成三维人脸模型,构建不同表情。 2. 利用手绘草图合成二维人脸素描。
	\begin{itemize}
		\item 对于三维人脸合成,基于人脸特征点,对于二维人脸合成,基于图像块特征。
		\item 使用基于局部坐标编码的耦合字典学习构建模型。
	\end{itemize}

\datedsubsection{\textbf{3. 视觉显著性相关算法研究}\ \textit{课题研究项目}}{2014年6月 -- 2015年3月}
\role{}{工作主页: https://www.ece.nus.edu.sg/stfpage/eleqiz/scene.html}

\end{onehalfspacing}

\begin{onehalfspacing}
内容:研究自然图像的视觉显著性。构建模型,预测人眼在自由观测图像时的关注点。
\begin{itemize}
  \item 提取颜色,灰度,梯度, 以及语义级别信息作为图像重要特征。
  \item 使用SVM建立视觉显著性预测模型,有效预测图片中的人眼关注点。
\end{itemize}
\end{onehalfspacing}




\datedsubsection{\textbf{4. 视觉显著性数据可视化}\ \textit{个人项目}}{2016年5月 -- 2016年7月}
\begin{onehalfspacing}
	内容:利用三维可视化方法对视觉显著性模型进行可视化分析。
	\begin{itemize}
		\item 利用视觉显著性数据构建三维可视化系统,方便分析与教学。
		\item 使用Three.js实现。
	\end{itemize}
\end{onehalfspacing}

% Reference Test
%\datedsubsection{\textbf{Paper Title\cite{zaharia2012resilient}}}{May. 2015}
%An xxx optimized for xxx\cite{verma2015large}
%\begin{itemize}
%  \item main contribution
%\end{itemize}

\section{\faCogs\ IT 技能}\small
% increase linespacing [parsep=0.5ex]
\begin{itemize}[parsep=0.5ex]
  \item 编程语言: Python == Matlab > C++,  平台: Linux > Windows,  办公: \LaTeX, Office
  \item Github: http://Github: github.com/Yanakz
\end{itemize}

\section{\faHeartO\ 获奖情况}\small
\datedline{浙江工业大学2015年研究生国家奖学金}{}
\datedline{浙江工业大学2015年信息工程学院三好研究生}{}


%% Reference
%\newpage
%\bibliographystyle{IEEETran}
%\bibliography{mycite}
\begin{thebibliography}{5}\small
	
\bibitem{1}
Haoran Liang, Guodao Sun and Ronghua Liang, 
\emph{``Looking into Saliency	Model via Space-Time Visualization"},
IEEE Transactions on Multimedia (T-MM), 2016. Acceptance.
	
\bibitem{2}
Haoran Liang, Ronghua Liang, et.al.,
\emph{``Coupled Dictionary Learning for Detail Enhanced 3-D Facial Expression Synthesis"},
IEEE Transactions on Cybernetics (T-CYB), 2015.		


\bibitem{3}
Haoran Liang, Mingli Song and Ronghua Liang
\emph{``Face Sketch Synthesis based on Rough Drawing"},
Journal of Computer Aided Design and Computer Graphics (JCAD\&CG), 2014.

\bibitem{4}
Haoran Liang, Mingli Song, et.al.,
\emph{``Personalized 3-D Facial Expreesion Synthesis based on Landmark Constraint"},
Asia-Pacific Signal and Information Processing Association (APSIPA), 2013.
\end{thebibliography}
\end{document}
